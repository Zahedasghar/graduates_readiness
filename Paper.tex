\documentclass[]{elsarticle} %review=doublespace preprint=single 5p=2 column
%%% Begin My package additions %%%%%%%%%%%%%%%%%%%
\usepackage[hyphens]{url}

  \journal{Business School} % Sets Journal name


\usepackage{lineno} % add

\usepackage{graphicx}
%%%%%%%%%%%%%%%% end my additions to header

\usepackage[T1]{fontenc}
\usepackage{lmodern}
\usepackage{amssymb,amsmath}
\usepackage{ifxetex,ifluatex}
\usepackage{fixltx2e} % provides \textsubscript
% use upquote if available, for straight quotes in verbatim environments
\IfFileExists{upquote.sty}{\usepackage{upquote}}{}
\ifnum 0\ifxetex 1\fi\ifluatex 1\fi=0 % if pdftex
  \usepackage[utf8]{inputenc}
\else % if luatex or xelatex
  \usepackage{fontspec}
  \ifxetex
    \usepackage{xltxtra,xunicode}
  \fi
  \defaultfontfeatures{Mapping=tex-text,Scale=MatchLowercase}
  \newcommand{\euro}{€}
\fi
% use microtype if available
\IfFileExists{microtype.sty}{\usepackage{microtype}}{}
\bibliographystyle{elsarticle-harv}
\ifxetex
  \usepackage[setpagesize=false, % page size defined by xetex
              unicode=false, % unicode breaks when used with xetex
              xetex]{hyperref}
\else
  \usepackage[unicode=true]{hyperref}
\fi
\hypersetup{breaklinks=true,
            bookmarks=true,
            pdfauthor={},
            pdftitle={Challenges for future of universities and jobs},
            colorlinks=false,
            urlcolor=blue,
            linkcolor=magenta,
            pdfborder={0 0 0}}
\urlstyle{same}  % don't use monospace font for urls

\setcounter{secnumdepth}{0}
% Pandoc toggle for numbering sections (defaults to be off)
\setcounter{secnumdepth}{0}


% tightlist command for lists without linebreak
\providecommand{\tightlist}{%
  \setlength{\itemsep}{0pt}\setlength{\parskip}{0pt}}


% Pandoc citation processing
\newlength{\cslhangindent}
\setlength{\cslhangindent}{1.5em}
\newlength{\csllabelwidth}
\setlength{\csllabelwidth}{3em}
\newlength{\cslentryspacingunit} % times entry-spacing
\setlength{\cslentryspacingunit}{\parskip}
% for Pandoc 2.8 to 2.10.1
\newenvironment{cslreferences}%
  {}%
  {\par}
% For Pandoc 2.11+
\newenvironment{CSLReferences}[2] % #1 hanging-ident, #2 entry spacing
 {% don't indent paragraphs
  \setlength{\parindent}{0pt}
  % turn on hanging indent if param 1 is 1
  \ifodd #1
  \let\oldpar\par
  \def\par{\hangindent=\cslhangindent\oldpar}
  \fi
  % set entry spacing
  \setlength{\parskip}{#2\cslentryspacingunit}
 }%
 {}
\usepackage{calc}
\newcommand{\CSLBlock}[1]{#1\hfill\break}
\newcommand{\CSLLeftMargin}[1]{\parbox[t]{\csllabelwidth}{#1}}
\newcommand{\CSLRightInline}[1]{\parbox[t]{\linewidth - \csllabelwidth}{#1}\break}
\newcommand{\CSLIndent}[1]{\hspace{\cslhangindent}#1}




\begin{document}


\begin{frontmatter}

  \title{Challenges for future of universities and jobs}
    \author[Quaid-i-Azam University, Islamabad]{Zahid Asghar\corref{1}}
   \ead{zasghar@qau.edu.pk} 
      \address[Quaid-i-Azam University]{School of Economics}
    
  \begin{abstract}
  Pakistan should fear more from lack of preparation to new challenges
  posed by digital revolution and automation of the work and covid-19
  pandemic disruption not than China/India/Bangladesh/Vietnam for
  capturing economic markets. Future of jobs and educational
  institutions are very uncertain. This study is aimed at how digital
  revolution, new technologies, artificial intelligence may lead to have
  major disruption in future jobs. It has been explored what possible
  approaches can be adopted and how our universities can tap those
  opportunities to put our burgeoning youth on a learning path so they
  can remain on job by adopting a culture of continuous learning.
  Opinion survey conducted from last year university students and
  MPhil/PhD students indicate element of worriness, lack of awareness of
  future uncertainties and relatively more focus on hard work than soft
  skills. Absence of career couseling and right places to get relevant
  skills also came out as some major factors for students poor readiness
  for market besides their poor performance in academics. Universities
  business as usual approach seems a complete disconnect for their
  readiness to meet the challenges of the 3rd decade of the 21st
  century. The future of both higher education and jobs nature is
  uncertain, and many potential futures exists. \textbf{Keywords: Future
  jobs, Future of Universities, Time vs Learning goals}
  \end{abstract}
  
 \end{frontmatter}

==========================

\hypertarget{introduction}{%
\subsubsection{Introduction}\label{introduction}}

Tens of millions of youth future and economic growth of Pakistan are
dependent on the type of learning ecosystem we are going to have. There
are differing degrees of inequalities including skills. The development
requires certain kind of skills that come with a formal education. This
is also called in economics jargon: skill-biased technical change. In
addition, there are informal know-how : tacit skills which develop when
one lives among other skilled workers. There is need for significant
effort and deliberate practice to think expansively to know what steps
are required for uncertain future work. If we will invest in developing
a learning ecosystem, our burgeoning youth will be equipped with
relevant skills to do well in the jobs of tomorrow. There is increasing
skilling gap in almost all of the occupations ranging from jobs of car
mechanic to digital services providers. Academia has not been out of its
comfort zone and trying to produce research which helps it to reach
higher grade instead of focusing on imparting skills which are in high
demand. Companies are not investing in training to develop requisite
skills. Many of these companies prefer to buy talent instead of building
talent with high level of professionalism. This has seriously
deteriorated the capacity to cope with challenges posed in the third
decade of the 21st century. Digital revolution and Covid-19 pandemic
disruption has exposed an ever increasing skilling gap at the globe, in
general and in a developing country like Pakistan, in particular.
Simultaneously, this has posed serious challenges to universities to
continue their business as usual. Universities are at the most trying to
build themselves to attain status as opposed to broad set of
institutional outcomes having utility for the society. To get their
improved ranking, they are rejecting more students than accepting. How
universities can transform to have inclusive learning and using
technological changes can provide education to all at a very low cost.
This changing role of universities will create a demographic dividend
for Pakistan which may play an active role for economic prosperity.
Changing university education by having full time in person to digital
emersion for online students, part time learning for workers and courses
for people of every age will help to have opportunities to excel now and
in future. This study contributes to reimagining university education to
tap opportunities created by covid-19 disruption and enabling graduates
to learn the process of getting skills. Future jobs skills are hardly
known to anyone. Traditional university education system is misaligned
with current job market and simply adding technological learning
management system on it will not resolve the issue.There is a huge gap
to bring educational system at par with emerging needs of the hour.
Simple singular narrative of making universities at par with the
requirements of the 3rd decade of the 21st century will not suffice
going forward. To face complexification of learning needs, there is need
for re-thinking the entire learning paradigm. There is need to switch
from one time learning mode for a given number of years to life long
learning mode. Jobs will now not be secured for life time, therefore,
there is need to provide part time opportunities. Nevertheless, these
challenges also offer new opportunities which will lead to convergence
for those who will get maximum benefits fo the use of digital tools
effectively. Life expectancy has increased over time and it is estimated
that it will increase further till the end of this century. So working
life will now be about 50 to 60 years in lower middle income countries
and maybe more in upper income countries. One has to keep on learning
for a long period of one's life. Secondly in developing countries, a
very small percent of population makes to tertiary education. There is
an urgent need for universities to introduce part-time studentship
opportunities to enable graduates to upgrade their skills and those
willing to earn part-time degree may get a chance. Use of educational
technology alone in universities will provide market-oriented skills is
largely inaccurate assumption. Our traditional educational system which
is misaligned to the needs of the students and society will deliver no
good by just adding any Learning Modular System. Main objective of this
study is to highlight that with penetration of artificial intelligence,
machine learning, big data and other digital techonologies has posed
serious challenges to those whose jobs are performed in a routinely
manner. Secondly, how universities can play a role and what will be
future perspective of universities after online learning platform has
become fully functional. Thirdly, what kind of learning ecosystem is
required where learning outcomes will be fixed while making time a
variable unlike practice so far in vogue where time is fixed and
learning is a variable. After introduction, it is discussed that the
challenges posed by digital technologies, artificial intelligence, big
data among others and how will it affect future jobs. Most of Pakistani
universities are suffering from serious financial crisis besides serving
a very small fraction of students at tertiary education. What is future
of universities without digital emersion and will physical universities
continue to operate after 10 years from now. Data collected on future
job challenges summary will help us to understand how our youth is aware
of these challenges and what is its understanding about future emerging
challenges with possible skill gaps.

\hypertarget{future-of-jobs-and-skills}{%
\subsubsection{Future of jobs and
skills}\label{future-of-jobs-and-skills}}

Future of jobs is uncertain due to rapid changes in the past 50 years in
all spheres of life all over the globe. Due to medical and technological
advances, life expectancy has increased from 30 years to 70 years and 80
plus years in the developed world. With increased life expectancy and
rapidly evolving technological changes, future of jobs has become
uncertain. Many jobs which are available were not there and many more
will emerge which are not available now. What kind of skills and
education should be imparted is not known. As a result, universities
role has also changed. Instead of imparting certain kind of skills,
universities have to think in learning ecosystem and have to strive to
enable students to learning path so that after completion of degrees,
students can learn at their own. There is urgent need to change learning
paradigm from enabling students to learn once mode to a continuous habit
of learning. Similarly, during the next 5 years, many new jobs will
appear. How can an individual or society meet this challenge? According
to Sohail Inayatull, often organization think when some disruption
happens or they miss something. Futuristic thinking means that working
on problems before some structural change makes one redundant. According
to him, often people think about the future, it is out there- Robotic,
Space travel, etc. Future is not like an empty space, it is like the
past. It is an active aspect of the present and thinking about the
future is to as change today.

\hypertarget{education-through-exploration}{%
\subsubsection{Education through
exploration}\label{education-through-exploration}}

Portable supercomputers somehow attached to the human being is now a
transformative thing. It's going to be the same as clothing,
electricity, refrigerators, and running water, and toilets. According to
Global Freshman Academy : core knowledge, advancing ideas allot for that
is still there. The future of higher education is uncertain and many
potential futures exists. Things are moving faster and faster. What
worked 10-20 years ago is not going to work 3 years from now or 5 years
from now. We have to be very systemic about sustainability and
technology when we think that in order to make a really affordable,
accessible, quality learning experience, we have to prepare for growth
ahead of time. Lifelong learning is important because education is going
to change as things are moving faster, new technologies are emerging,
skills to use those new technologies are in high demand. Therefore, it
is not viable that a 4-year education, once in a lifetime, will serve
one for the next 40/50 years. There are 100s of courses available for
online learning but its important that universities design their own
courses as per local needs. These must be in English language but while
teaching, there is no harm if local language can be used or can be
bilingual. Local case studies will help to engage students better.

\hypertarget{continuous-learning-ecosystem}{%
\subsubsection{CONTINUOUS LEARNING
ECOSYSTEM}\label{continuous-learning-ecosystem}}

Universities need commitment to the promotion of lifelong learning
through its academic programs and promotion of good citizenship through
community-based learning process. People's whom jobs will be abandoned
don't have skills for new jobs to be created. It is not feasible for
most of such people to get education at campus even if there are such
upskilling programs. Majority of the jobs are linked with soft skills
for which physical labs are not required. Therefore, universities need
to offer online learning programs to retrain existing employers and for
upskilling of their jobs. Academic institutions inertia is a real but
its not alone academic institutions, but onus is also on companies just
as much as it is on higher education to meet the challenges of changing
nature of jobs. The reskilling crisis is emerging very rapidly and the
dilemma is Pakistan is niether ready to realize this crisis nor any
preparation in near future to bridge this ever increasing reskilling
crisis. Pakistani academia, companies and governments have long been
trained in a way where issues can be resolved with more hard-work with
mediocre skills rather increasing productivity with better skills and
smart work instead of hard work. Survey results Complexity of the 21st
century challenges required new skills and CONTINUOUS LEARNING approach.
Now, neither universities nor companies are thinking hard to devise
measures to deal with this complexification. Harvard Business School
survey revealed that most businesses leader prefer to invest in
technology rather than overhaul complex human capital management. Focus
is more buying a talent instead of building a talent. Culture of
retaining and attracting talent is absent. No steps are taken to
accelerate a new and transformative human capital development agenda for
the work of future. \textbf{``And now that future is our present of
work.''}

\hypertarget{knowledge-economy}{%
\subsubsection{Knowledge Economy}\label{knowledge-economy}}

There is lot of lip service to knowledge economy without realising what
it is and what its demands are, and how can a nation prepares itself for
the same. Due to increased life expectancy over time, it is hard to
imagine a straight line from education to work and, finally, retirement.
2 or 4 or even 6 years college front-loaded at the beginning of 80 to
100 years life seems inadequate. There is a need a for paradigm shift
from default mental model Learn, Earn, Rest to Learn, Earn, Learn,
Earn,\ldots{} for 10 to 12 jobs changes in one's life time. Learning and
work have become inseparable, and it is knowledge economy or continuous
learning. There is complete disconnect what is supplied and what is the
demand in the market. I restrict myself here to role of higher education
in meeting this challenge of knowledge economy. Higher education system
is stuck in first transition from young adulthood to work force. There
is debate whether universities should produce graduates with skills
matching with jobs or graduates should follow a broader general skill.
There are strong arguments on both sides. But research shows that
graduates who start their career path at lower level than their
qualification are highly likely to serve at lower level except few
disciplines. Secondly, surveys at entry level indicate that majority of
them join college for getting skills to get job in the market. Thirdly,
50 to 70 years work life demands that job skills are must for a
graduate. It is not true to think college education against workforce
training. As economist Anthony P. Carnevale writes: ``The inescapable
reality is that ours is a society based on work. Increasing the economic
relevance of education should, if done properly,extend the ability of
educators to empower Americans to work in the world, rather than retreat
from it.'' Supply-Demand Mismatch has become a dominant issue. There is
a need to have symbiotic relationship among learners, learning providers
and employers. According to LearnLaunch estimates for 2015-2018, more
than 240 new companies secured funding to address supply-demand mismatch
issues, workplace competencies, technical skills, and formal and
informal training making up a market place of workforce in excess of
billion.

\hypertarget{career-counseling-and-support}{%
\subsubsection{Career Counseling and
Support}\label{career-counseling-and-support}}

There are lot of online learning opportunities but many often ask that
they dont have access to trustworthy sources from where they can get an
idea from which source to learn. Unlike consumer common products used in
Pakistan, there are not much reviews about learning platforms. Some
complains that online learning training are either below quality or fake
and charge hefty amounts to participants. Workers often ask how to make
sense of their options but there is not sufficient information available
on the matter. Potential consumers of online education encounter a black
box. A survey from university graduates mentions that they are
performing poor in studies because no right counseling was provided to
them when they joined BS-graduate at college level. Similar gaps exists
between learners, learning providers and employers Neither side
understands clearly what the other sides need. Adult learners need
guidance and are in need of mentors. Most of them are unable to move on
a learning curve at their own, therefore, they need guidance. With human
help, adult learners can make their online learning more effective.
Learning and upskilling alone are not enough. Codding apprenticeship,
legal services, food stamps, health care, employment assistance It is
becoming difficult for youth to spare mental energies to creativity due
to increasing pressure of earning livelihood. Mental energies not free
to put but survival. Despite a number of free learning opportunities,
how can they fit in extra learning along with day-to-day struggle to
survive? Guidelines to reskilling/upskilling are not available, and it
is not possible for workers to improve their skills and once they get
out of job, they hardly find any working opportunity afterward due to
change in required skills.

Table/Graph

Networking and social capital are just like assets as one can utilize
these to perform many assignments efficiently. Office staff needs
continuous support and their managers need training as well. But there
is a dominant culture of management in Pakistan, so management and other
leaders often say ``Oh, I have managed people before, I dont need
training.'' Continuous support from education leaders/managers have
significant long-term benefits for new workers as they prepare for
success in the new economy.

\hypertarget{future-of-universities}{%
\subsubsection{Future of Universities}\label{future-of-universities}}

According to Transformation 2050 book, due to demographic transitions,
digitalization, globalization and recent pandemic covid-19 disruption,
higher education across the globe is in process of massive changes.
Unfortunately, Pakistani university culture is like a factory model or
the forced-feed one which has reached its limits but unfortunately our
higher education leaders don't know how to move forward. Leadership
should not have vision and courage to lead to future but also make this
process participative and inclusive. {[}Transformation 2050: The
Alternative Futures of Malaysian Universities By Sohail Inayatullah and
Fazidah Ithnin (with contributing chapters by Azhari-Karim, Ellisha
Nasruddin, Reevany Bustami, Ivana Milojević) USIM Press, Universiti
Sains Islam Malaysia, Bandar Baru Nilai, Negeri Sembilan, 2018{]} Though
it is claimed that skills are given more weightage than degrees but
still even in country like United States a college degree signal sends a
strong signal and current working learners are not preferred over those
who carry degrees. Instead of focusing on degrees, it is important to
have precise and relevant education tailored to the needs of society,
employers and learners. Right skills at the right time in the right
pathways are needed. Can one think where universities offer part time
degrees in Pakistan so that continuous working learners are not left
behind just because of a degree signal. For this one has to redevise
course structure as there are a number of courses like Pakistan Study,
English, Islamic Study and many other general courses required to
complete a degree. Such courses are not required for those who are in
jobs and need degrees relevant to their skills to move higher ladder in
their career. But our universities dont have any program so these can
serve those un-served ones. Signaling power of degree is so strong that
there is no equal in job market to it. \textgreater{} Dr.~Michael Crow
``Universities should be great contributors to society rather
universities to be great contributors to themselves. Each university is
trying to be as good as Harvard, Berkley and Yale. Many may achieve the
goal and what else is to achieve if its done.

Building institutions around the sole attainment of status as opposed to
a broader set of institutional outcomes, which would include - status on
the list - achievement on the list but also achieve somewhat impact.
Emergence of new kind of institutions is needed with having digital
emersion. We are all about creating, discovering, analyzing,
synthesizing, storing and transferring knowledge. That's what we do
between and among generations.

\hypertarget{problem-based-learning}{%
\subsubsection{Problem Based Learning}\label{problem-based-learning}}

21st century 3rd decade learning pathways must be different and move
away from memorization and standardized testing to problem-based
learning. Hands-on experimentation is required where learners engage in
productive struggle, persist and learn just in time-specific
disciplinary concepts. Secondly, there is not a single problem in real
life which can be solved in isolation. Solution to problem requires
multidisciplinary approaches but unfortunately our university system is
providing degrees in complete isolation to each other and even within
same faculty, departments/schools are working in silos. Use talent
development as a blueprint for varying the problems and building the
solutions that are happening in our community everyday. There is need
that policymakers, workforce, learning providers and employers
understand skill gaps to work out how to close those gaps through design
and development of well-established and more precise learning pathways.

\hypertarget{exploratory-data-analysis}{%
\subsubsection{Exploratory Data
Analysis}\label{exploratory-data-analysis}}

We have conducted two surveys. In one of the survey, main question was
how career counseling contribute to students' academic performance and
in their career decision making. In second survey, questions regarding
future job challenges and soft skills required to meet those challenges
were asked mainly from final year graduate and research postgraduate
students. Additionally, some results from a survey conducted from 14000
free lancing individuals are reported. These individuals are working as
free lancers both in national and international market after getting
digital skills from Digiskills, Ignite, Ministry of Finance, Islamabad.
Results of first two surveys are not completely random and may suffer
from high response bias. Therefore, confirmatory data analysis is not
performed, and results are mainly based on exploratory data analysis.

\hypertarget{usage}{%
\subsubsection{Usage}\label{usage}}

Once the package is properly installed, you can use the document class
\emph{elsarticle} to create a manuscript. Please make sure that your
manuscript follows the guidelines in the Guide for Authors of the
relevant journal. It is not necessary to typeset your manuscript in
exactly the same way as an article, unless you are submitting to a
camera-ready copy (CRC) journal.

\hypertarget{functionality}{%
\subsubsection{Functionality}\label{functionality}}

The Elsevier article class is based on the standard article class and
supports almost all of the functionality of that class. In addition, it
features commands and options to format the

\begin{itemize}
\item
  document style
\item
  baselineskip
\item
  front matter
\item
  keywords and MSC codes
\item
  theorems, definitions and proofs
\item
  lables of enumerations
\item
  citation style and labeling.
\end{itemize}

\hypertarget{front-matter}{%
\section{Front matter}\label{front-matter}}

The author names and affiliations could be formatted in two ways:

\begin{enumerate}
\def\labelenumi{(\arabic{enumi})}
\item
  Group the authors per affiliation.
\item
  Use footnotes to indicate the affiliations.
\end{enumerate}

See the front matter of this document for examples. You are recommended
to conform your choice to the journal you are submitting to.

\hypertarget{bibliography-styles}{%
\section{Bibliography styles}\label{bibliography-styles}}

There are various bibliography styles available. You can select the
style of your choice in the preamble of this document. These styles are
Elsevier styles based on standard styles like Harvard and Vancouver.
Please use BibTeX~to generate your bibliography and include DOIs
whenever available.

Here are two sample references: (Dirac, 1953; Feynman and Vernon Jr.;
1963).

\hypertarget{references}{%
\section*{References}\label{references}}
\addcontentsline{toc}{section}{References}

\hypertarget{refs}{}
\begin{CSLReferences}{1}{0}
\leavevmode\vadjust pre{\hypertarget{ref-Dirac1953888}{}}%
Dirac, P.A.M., 1953. The lorentz transformation and absolute time.
Physica 19, 888--896.
doi:\href{https://doi.org/10.1016/S0031-8914(53)80099-6}{10.1016/S0031-8914(53)80099-6}

\leavevmode\vadjust pre{\hypertarget{ref-Feynman1963118}{}}%
Feynman, R.P., Vernon Jr., F.L., 1963. The theory of a general quantum
system interacting with a linear dissipative system. Annals of Physics
24, 118--173.
doi:\href{https://doi.org/10.1016/0003-4916(63)90068-X}{10.1016/0003-4916(63)90068-X}

\end{CSLReferences}


\end{document}
